\documentclass{article}
\usepackage[utf8]{inputenc}
\usepackage{geometry}
\geometry{
	hmargin=2.1cm,
	vmargin=1.85cm,
	a4paper,
	%showframe,
}
\usepackage{paracol}
\columnratio{0.55,0.45}
\setlength\columnsep{0.03\textwidth}

\usepackage{enumitem}

\usepackage[usenames,svgnames]{xcolor}
\usepackage{colortbl}
\usepackage{color}
\definecolor{slateblue}{rgb}{0.17,0.22,0.34}
\definecolor{headings}{rgb}{0.415,0.631,0.584}

\definecolor{text}{HTML}{2b2b2b}
\definecolor{shade}{HTML}{F5DD9D}
\definecolor{linkcolor}{rgb}{0.862,0.572,0.592}
\color{text}
\newcommand{\chline}{\arrayrulecolor{headings}\hline\arrayrulecolor{headings}}
\newcommand{\crule}[2]{\textcolor{headings}{\rule{#1}{#2}}}

\usepackage{hyperref}
\hypersetup{colorlinks, breaklinks, urlcolor=linkcolor, linkcolor=linkcolor, linktoc=all}

\makeatletter
\renewcommand{\maketitle}{
\bgroup\setlength{\parindent}{0em}
  {\scshape\huge\color{headings} \@title}
  \vspace*{6pt}
  {\color{headings}\hrule depth 0.8pt \relax}
  \vspace*{6pt}
  \hspace*{2em}{\@author \hfill \@date\hspace*{2em}}\\[1em]
\egroup
}
\makeatother

\usepackage[nobottomtitles*]{titlesec}
\renewcommand{\bottomtitlespace}{0.1\textheight}
\titleformat{\section}
    {\color{headings}\scshape\Large\raggedright}
    {\thesection}
    {1em}
    {}
    [\color{headings}\titlerule]
\titlespacing{\section}
    {0pt}
    {0pt}
    {1em}
\titleformat{\subsection}
    {\color{headings}\scshape\large\raggedright}
    {\thesubsection}
    {1em}
    {}
\renewenvironment{abstract}
    {}
    {
    \begin{quote}
        \noindent
    \end{quote}
    }

\usepackage{tcolorbox}
\usepackage{listings}
\usepackage{listings-golang}
\lstset{
    frame=single,
    frameround=tttt,
    xleftmargin=1.3em,
    xrightmargin=1.3em,
    aboveskip=1mm,
    belowskip=3mm,
    backgroundcolor=\color{gray!5},
    showstringspaces=false,
    columns=flexible,
    basicstyle={\small\ttfamily},
    numbers=none,
    breaklines=true,
    breakatwhitespace=true,
    tabsize=4,
    keywordstyle=\color{cyan},
    stringstyle=\color{green!80!black},
    commentstyle=\color{brown!60},
    language=Golang,
}

\usepackage{environ}

\title{\scshape Go notes for TIC-80}
\author{Fergus Baker}

\newcommand{\tinygo}{\texttt{tinygo}\,}
\newcommand{\WASM}{\texttt{WASM}\,}
\newcommand{\fullref}[1]{\hyperref[{#1}]{\ref{#1} \nameref{#1}}}

\begin{document}
\maketitle

\begin{abstract}
    Notes for using Go with TIC-80 by targeting WASM with \tinygo.
\end{abstract}

\begingroup
\color{linkcolor}
\tableofcontents
\endgroup
\vspace{2em}

% \pagebreak


\section{Language overview}

\subsection{Types}

See also \fullref{s_interfaces} concerning \textit{type assertions}.
\\

\columnratio{0.4}
\begin{paracol}{2}

\noindent Go supports the primitives listed on the right hand side.

\switchcolumn
\begin{lstlisting}
// integral
int, int8, int16, int32, int64
uint, uint8, uint16, uint32, uint64, uintptr
// floating point
float32, float64
complex64, complex128
//other
string, bool
// aliases
byte // uint8
rune // int32 for unicode
\end{lstlisting}

\switchcolumn*
\noindent Types may be converted to some new type \lstinline{T} using \lstinline|T(v)| .
\\
\switchcolumn
\switchcolumn*
\noindent Type aliases may be defined using the \lstinline|type| keyword.

\switchcolumn
\begin{lstlisting}
type MyFloat float64
\end{lstlisting}

\end{paracol}

\subsection{Declarations}

\columnratio{0.45}
\begin{paracol}{2}
\noindent Variables may either be explicitly typed or inferred. Constants may be inferred, and support arbitrary precision until coerced.

\switchcolumn

\begin{lstlisting}
// explicit type
var i int
// explicit type initialized
var i int = 1
// implicit type
i := 1
// constant
const Pi = 3.14
\end{lstlisting}


\switchcolumn*

\noindent Pointers are either declared to point to an array or a variable. References can be taken with the \lstinline{&} operator.

\switchcolumn

\begin{lstlisting}
// pointer to array
var a []int
x = a[1]
// pointer to type
var p *int 
// dereference
x = *p
// reference
q := &x
\end{lstlisting}

\switchcolumn*

\noindent The allocation primatives are \lstinline|make| and \lstinline|new|, and apply to different types. \lstinline|new| allocates and zeros memory, returning a \textbf{pointer} \lstinline|*T|. The keyword \lstinline|make| is reserved only for slices, maps, and channels, and \textbf{does not} return a pointer.
\\

\switchcolumn

\begin{lstlisting}
// p is *MyCustomType
p := new(MyCustomType)
// v is MyCustomType
var v MyCustomType
\end{lstlisting}

\end{paracol}


\subsection{Structs}

\columnratio{0.45}
\begin{paracol}{2}
Structs are a collection of fields, which are \lstinline|{}| initialized. Pointers to structs have a free level of indirection, thus \lstinline{(*p).x} is the same as \lstinline{p.x}. Uninitialized fields are implicitly zero.
\switchcolumn    


\begin{lstlisting}
type Vertex struct {
    X int
    Y int
}
v1 := Vertex{1, 2}
v2 := Vertex{Y: 2} // X implicitly 0
\end{lstlisting}

\switchcolumn*

\noindent Go implements the concept of \textbf{constructors as factories}, which is conventionally the name of the struct prefixed with \lstinline|New|.

\switchcolumn
\begin{lstlisting}
func NewVertex(x, y int) Vertex {
    return NewVertex{x, y}
}
\end{lstlisting}

\end{paracol}

\subsection{Functions}

\noindent \textbf{Functions} must be explicitly typed and support multiple (named) return types.
Function are first class citizens and may be assigned to variables. Functions support closure capture.
\\

\begin{lstlisting}
// function type
func(int32, int32) int32

// anonymous: a and b have same type
adder := func(a, b int32) int32 {return a + b}

// single return type
func foo(a int32, b int32) int32 {
    return a + b
}

// multiple returns
func mfoo(a, b int32) (int32, int32) {
    return a, b
}

// multiple returns named
func bar(a int32, b int32) (out1 int32, out2 int32) {
    out1 = a
    out2 = b
    return
}

// closure capture
func adder() func(int) int {
    sum := 0
    return func(x int) int {
        sum += x
        return sum
    }
}
\end{lstlisting}

\begin{paracol}{2}
\noindent The address of a \textbf{local variable} may be returned without issue: the storage of a variable survives the function context. Referencing an r-value \textbf{allocates a new} instance.

\switchcolumn

\begin{lstlisting}
func NewFile(fd int, name string) *File {
    if fd < 0 {
        return nil
    }
    // new instance each time it is called
    return &File{fd: fd, name:name}
}
\end{lstlisting}

\switchcolumn*
\end{paracol}

\subsection{Methods}

\begin{paracol}{2}
Methods may be defined on \textbf{types} (such as custom structs). Methods have a special \textbf{receiver} argument.

\switchcolumn
\begin{lstlisting}
func (v Vertex) Abs() float64 {
    return math.Sqrt(v.X * v.X + v.Y * v.Y)
}
// invocation
v.Abs()
\end{lstlisting}

\switchcolumn*

\noindent For methods to be mutating they must be declared with \textbf{pointer receivers}.

\switchcolumn
\begin{lstlisting}
func (v *Vertex) Scale(f float64) {
    v.X = v.X * f
    v.Y = v.Y * f
}
\end{lstlisting}


\end{paracol}

\subsection{Arrays}

\begin{paracol}{2}

Go arrays are 0 indexed. Fixed size arrays are declared with \lstinline{[n]T} syntax. Slices are dynamically sized references to arrays, declared with \lstinline{[]T}.
Slices may be literal.

When slicing, the bounds are implicitly the start and end if excluded.
\switchcolumn    

\begin{lstlisting}
// array of 10 ints
var a [10]int
// slice to 3 ints and indices 1, 2, 3
b := a[1:4]
// slice literal
c := []bool{false, false, false}
\end{lstlisting}


\switchcolumn*
\noindent The \lstinline{len} of a slice is the number of elements it contains, whereas the \lstinline{cap} is the number of elements in the underlying array.

\lstinline{nil} slices are slices with length and capacity equal to 0.

\switchcolumn    

\begin{lstlisting}
s := []int{2, 3, 5, 7, 11, 13}
// len=6 cap=6 [2 3 5 7 11 13]
s = s[:0]
// len=0 cap=6 []
s = s[:4]
// len=4 cap=6 [2 3 5 7]
s = s[2:]
// len=2 cap=4 [5 7]
\end{lstlisting}


\switchcolumn*
\noindent Slices may be \textbf{dynamically allocated} with the \lstinline|make| function, which allocates and zeros out an array.\\

\switchcolumn

\begin{lstlisting}
// len 5, cap 5
a := make([]int, 5)
// len 0, cap 5
b := make([]int, 0, 5)
\end{lstlisting}


\end{paracol}

\subsection{Maps}
\begin{paracol}{2}

A map is a key-value store. The zero value of a map is \lstinline|nil|. Maps are dynamically allocated and must be initialized with \lstinline|make|.
    
\switchcolumn
\begin{lstlisting}
// variable declaration
var m = map[string]int
// init
m = make(map[string]int)
\end{lstlisting}

\switchcolumn*
\noindent Maps may be \textbf{mutated} with the usual \lstinline|[]| syntax. When an entry is read, the map returns both the value an an \lstinline|ok| boolean. If the key is not in the map, the value is a zero and \lstinline|ok| is \lstinline|false|.

\switchcolumn
\begin{lstlisting}
// add entry
m["Hello World"] = 42
// read entry
value, ok := m["Hello World"]
// remove
delete(m, "Hello World")
\end{lstlisting}
\switchcolumn*
\noindent Map literals may also be declared.

\switchcolumn
\begin{lstlisting}
// map literal
ml = map[string]int{
    "Hello": 13,
    "World": 12,
}
\end{lstlisting}

\end{paracol}

\subsection{Control flow}

See also \fullref{s_interfaces} concerning \textit{type switches} for control flow.
\\
\columnratio{0.5}
\begin{paracol}{2}
\noindent A \textbf{defer} statement defers the execution of a function until the surrounding function returns. Defers are executed in LIFO order.
\\
\switchcolumn


\begin{lstlisting}
defer fmt.Println("world")
fmt.Println("hello")
\end{lstlisting}



\switchcolumn*
\noindent The \textbf{for} loop has an initializer, a condition, and a post statement, with the initializer and post statement being optional.

\switchcolumn

\begin{lstlisting}
for i := 0; i < 10: i++ {
    // ...
}
\end{lstlisting}


\switchcolumn*

\noindent The \textbf{while} loops have the same syntax, though the semi colons may be dropped.

\switchcolumn

\begin{lstlisting}
// while
for i < 10 {
    // ...
}
\end{lstlisting}


\switchcolumn*
\noindent Infinite loops are created without any arguments.

\switchcolumn

\begin{lstlisting}
// infinite
for {
    // ...
}
\end{lstlisting}


\switchcolumn*

\noindent For loops may also be used with \textbf{range indexing}, implicitly enumerating the array or slice.

\switchcolumn

\begin{lstlisting}
var pow = []int{1, 2, 4, 8}
for i, v := range pow {
    // ...
}
\end{lstlisting}



\switchcolumn*

\noindent \textbf{If} statements have need not have brackets, and support optional capture initializers. The variable in the initializer only exists for the scope of the `if' block.

\switchcolumn

\begin{lstlisting}
if some_condition {
    // ...
} else {
    // ...
}
// with a capture init
if y := a + b; y < 10 {
    return y
}
\end{lstlisting}


\switchcolumn*

\noindent The \textbf{switch} statement can be used on any primitive. Cases do not have fall-through, and the cases need not be constants.

\switchcolumn

\begin{lstlisting}
switch os := runtime.GOOS; os {
case "darwin":
    // ...
case "linux":
    // ...
default:
    // ...
}
\end{lstlisting}


\switchcolumn*
\noindent In the example to the right, \lstinline{f()} is not invoked unless \lstinline{i != 0}.
A switch statement without a condition is the same as \lstinline{switch true}.

\switchcolumn

\begin{lstlisting}
switch i {
case 0: 
    // ...
case f():
    // ...
}
\end{lstlisting}

\end{paracol}

\subsection{Interfaces}\label{s_interfaces}

From what I can tell, the Go-ism for interface names is to end the interface with \lstinline|-er|, e.g. \lstinline|fmt.Stringer|.
\\

\columnratio{0.45}
\begin{paracol}{2}

Interfaces are \textbf{types} that define a set of method signatures. They may be thought of as a tuple of \lstinline|(value, type)|

\switchcolumn
\begin{lstlisting}
type Absoluter interface {
    Abs() float64
}
\end{lstlisting}
\switchcolumn*

\noindent The \textbf{nil interface} is the interface which implement no methods, such as primitives. It is declared with \lstinline|interface{}|
\\
\switchcolumn

\begin{lstlisting}
var i interface{}
\end{lstlisting}

\switchcolumn*

\noindent Variables of instance type may be declared and instantiated by any type which implements the interface.

\switchcolumn
\begin{lstlisting}
type SomeFloat float64
func (f SomeFloat) Abs() float64 {
    // ...
}
// instantiate interface
var a Absoluter
a = SomeFloat(1.0)
\end{lstlisting}

\switchcolumn*

\noindent Care must be taken when dealing with pointers. In this example, \lstinline|*Vector| \textit{does} implement the interface, but \lstinline|Vector| \textit{does not}.

\switchcolumn

\begin{lstlisting}
type Vector struct {
    X, Y float64
}
func (v *Vector) Abs() float64 {
    // ...
}
// ok
v := Vector{}
var a Absoluter = &v
// error
a = v 
\end{lstlisting}
\switchcolumn*

\noindent \textbf{Type assertions} may be used to access an interfaces concrete value. It returns a \lstinline|(value, ok)| tuple, where the \lstinline|ok| boolean denotes whether the type assertion was true. The syntax is \lstinline|t, ok := i.(T)| to assert type \lstinline|T|.

\switchcolumn
\begin{lstlisting}
var i interface{} = "Hello World"
// ok is true
s, ok := i.(string) 
// ok is false, v is 0.0
f, ok := i.(float64)
\end{lstlisting}
\switchcolumn*

\noindent Switches may also be used on interfaces as control flow with \textbf{type switches}.

\switchcolumn
\begin{lstlisting}
switch v := i.(type) {
case T:
    // here v has type T
case S:
    // here v has type S
default:
    // no match; here v has the same type as i
}
\end{lstlisting}

\switchcolumn*

\noindent Interfaces may \textbf{embed} or extend existing interfaces by including them in the definition.

\switchcolumn

\begin{lstlisting}
type A interface {
    GetName() string
}
// embeds A
type B interface {
    A
    SetValue(v int)
}
\end{lstlisting}

\switchcolumn*

\noindent Go differentiates between \textbf{basic} and \textbf{non-basic} interfaces, where basic interfaces may be entirely implemented and initialized, whereas non-basic interfaces are used primarily in \fullref{sec:generics}.

Non-basic interfaces are interfaces with \textbf{type-unions} (the pipe operator), or which embed other non-basic interfaces.

\switchcolumn

\begin{lstlisting}
// the above A and B are both basic
// the below C and D are non-basic
type C interface {
    int | int64 | float64
}
type D interface {
    C
    Content() string
}
\end{lstlisting}

\end{paracol}

\subsection{Errors}

\begin{paracol}{2}

The error type is an interfaces that implements the \lstinline|Error() string| method. 

\switchcolumn

\begin{lstlisting}
type error interface {
    Error() string
}
\end{lstlisting}

\switchcolumn*

Errors are normally returned in a tuple with the result. Errors are handled by testing \lstinline|err != nil|.

\switchcolumn

\begin{lstlisting}
i, err := strconv.Atoi("42")
if err != nil {
    // ...
    return
}
\end{lstlisting}

\switchcolumn*

\noindent Go also has \lstinline|panic| and \lstinline|recover|. The \lstinline|panic| keyword is reserved for errors that are ``unrecoverable'', and \lstinline|recover| is to recover from them. When a \lstinline|panic| is called, go immediately begins the stack unwind until it hits a \lstinline|recover|.

\lstinline{recover} \textbf{always returns} \lstinline|nil| unless called from a deferred function.

\switchcolumn

\begin{lstlisting}
func server(workChan <-chan *Work) {
    for work := range workChan {
        go safelyDo(work)
    }
}

func safelyDo(work *Work) {
    // deferred lambda
    defer func() {
        if err := recover(); err != nil {
            log.Println("work failed:", err)
        }
    }()
    do(work)
}
\end{lstlisting}

\end{paracol}

\subsection{Generics}\label{sec:generics}

\begin{paracol}{2}

Functions may accept \textbf{type parameters} for generics. This parameter appears in \lstinline|[]| before the functions arguments. The type parameter must fulfill a \lstinline|constraint|.
  
\switchcolumn
\begin{lstlisting}
// T must support == comparison
func Index[T comparable](s []T, x T) int {
    // ...
}

\end{lstlisting}
\switchcolumn*
\noindent Constraints may be defined through interfaces. The super-type of all interfaces is \lstinline|any|. The pipe syntax denotes type unions.

\switchcolumn

\begin{lstlisting}
type Number interface {
    int | int64 | float64
}
    
\end{lstlisting}

\end{paracol}

\subsection{Exports and imports}
\begin{paracol}{2}

Exported functions from a package must begin with a capital letter. Packages which make use of \lstinline{helloworld} may only refer to \lstinline{helloworld.Foo}.

\switchcolumn

\begin{lstlisting}
package helloworld
// not exported
func foo() {
    // ....
}
// is exported
func Foo() {
    // ...
}
\end{lstlisting}

\switchcolumn*

\noindent Go also uses \textbf{compiler directives} in the form of annotated functions with \textbf{comments} to export or import symbols at link-time.

\switchcolumn

\begin{lstlisting}
// link the symbol _start to the function Init
//go:linkname Init _start
func Init() 

// import a function symbol
//go:export
func add(x, y int) int

// export a function symbol
//go:export
func sub(x, y int) int {
    return x - y
}


\end{lstlisting}

\end{paracol}

\section{Goroutines and concurrency}

\noindent A \textbf{goroutine} is a lightweight thread that is managed by the runtime. The keyword \lstinline|go| is reserved for starting a new goroutine. They run in the same address space. The \lstinline|sync| package provides goroutine primatives.

\subsection{Channels}

\begin{paracol}{2}
Channels are a typed pipe for IO, and may be used to send or receive with the \textbf{channel operator}, i.e. \lstinline|<-|. Channels are by default blocking to allow goroutines to synchronize without explicit locking mechanisms.

\switchcolumn

\begin{lstlisting}
// initialize integer channel
ch := make(chan int)
// send 
ch <- value
// receive
v := <- ch
\end{lstlisting}

\switchcolumn*

\noindent Channels may be \textbf{buffered}, which means they have a fixed size and will result in a deadlock if trying to send to a full buffer.
\\

\switchcolumn
\begin{lstlisting}
// buffered channel with 100 elements
ch := make(chan int, 100)
\end{lstlisting}

\switchcolumn*

\noindent Channels may be \textbf{closed} to indicate no additional values will be sent. A second return argument indicates whether a channel is closed or not. It is not necessary to \textit{always} close channels, and should be used only to indicate no additional information.


\switchcolumn

\begin{lstlisting}
ch := make(chan int)
ch <- 10
// close channel
close(ch)
// ok == false since channel closed
v, ok := <- ch
\end{lstlisting}
\switchcolumn*

\noindent The \lstinline|range| keyword may be used in a for-loop to read all values from a channel until closed.

\switchcolumn
\begin{lstlisting}
// read until closed
for i := range ch {
    // ...
}
\end{lstlisting}
\switchcolumn*

\noindent The \lstinline|select| keyword is analogous to the \lstinline|switch| statement for waiting on multiple operations. The \lstinline|select| blocks execution until one of its cases can run, then executes that case. It is non-deterministic if multiple cases are ready simultaneously. The \lstinline|default| case is optional, and will run if no other cases are ready.

\switchcolumn

\begin{lstlisting}
select {
// will only run if ch not full
case ch <- x:
    x = x + 1
// will run if can read from recv
case <- recv:
    fmt.Println("received")
// optional default
default:
    fmt.Println("no operation)
}
\end{lstlisting}

\end{paracol}

\subsection{Example}

Below is a concurrent example for summing numbers in an array:\\

\begin{lstlisting}
package main

func sum(s []int, c chan int) {
    sum := 0
    for _, v := range s {
        sum += v
    }
    c <- sum
}

func main() {
    s := []int{7, 2, 8, -9, 4, 0}
    // make a channel for io between goroutines
    c := make(chan int)
    // spawn two goroutines that sum different parts of s
    go sum(s[:len(s)/2], c)
    go sum(s[len(s)/2:], c)
    // receive
    x, y := <-c, <-c 
}
\end{lstlisting}

\subsection{Mutexes}

\columnratio{0.5}
\begin{paracol}{2}

The \lstinline|sync.Mutex| (mutual exclusion) can be used when multiple goroutines need to access the same resource without worrying about race conditions. The standard mutex supports \lstinline|Lock| and \lstinline|Unlock| methods.

\switchcolumn

\begin{lstlisting}
mu := sync.Mutex()
mu.Lock() 
// practice is to defer unlocks
defer mu.Unlock()
\end{lstlisting}

\end{paracol}

\section{Common interfaces}

\subsection{Readers and writers}

\columnratio{0.45}
\begin{paracol}{2}
The \lstinline|io| package specifies a \lstinline|io.Reader| interface, which declares the \lstinline|Read| method.

\switchcolumn

\begin{lstlisting}
// prototype
func (T) Read(b []byte) (n int, err error)
// example
r := strings.NewReader("Hello, Reader!")
buffer := make([]byte, 8)
for {
    n, err := r.Read(buffer)
    // ...
    if err == io.EOF {
        break
    }
}
\end{lstlisting}


\end{paracol}

\section{Standard patterns}

\subsection{Errors and recovery}
\begin{paracol}{2}

\noindent Throw errors with \lstinline|panic| so that goroutines can handle and recover as needed.

\switchcolumn

\begin{lstlisting}
func (s *S) error(err string) {
    panic(Error(err))
}

func Runner(str string) (s *S, err error) {
    s = new(S)
    defer func() {
        if e := recover(); e != nil {
            // Clear return value.
            regexp = nil    
            // will re-panic cannot coerce
            err = e.(Error) 
        }
    }()
    // will panic if there is a call error
    return s.Call(str), nil
}
\end{lstlisting}
\end{paracol}

\subsection{Parallelism}

\begin{paracol}{2}
\noindent Goroutines can be used for concurrency by initializing e.g. multiple channels. 

\switchcolumn

\begin{lstlisting}
const numCPU = 4 // number of CPU cores

func (v Vector) DoAll(u Vector) {
    // buffered array
    c := make(chan int, numCPU) 
    fraction := len(v) / numCPU
    for i := 0; i < numCPU; i++ {
        go v.DoSome(
            // lower index
            i * fraction, 
            // upper index
            (i+1) * fraction, 
            u, 
            c
        )
    }
    // drain the channel.
    for i := 0; i < numCPU; i++ {
        // wait for one task to complete
        <-c 
    }
    // all done.
}
    
\end{lstlisting}

\switchcolumn*
\end{paracol}

\subsection{Enums}

\begin{paracol}{2}
Although Go does not have a concept of an enum, there is the builtin \lstinline|iota|, which is an automatically incrementing integer scoped to \lstinline|const| blocks with the initial value of 0. It can be used to quickly define monotonically increasing constants. \lstinline|iota| may also be used in expressions.

\switchcolumn

\begin{lstlisting}
const (
    A = 1
    B  = 2
    C  = 4
)
// becomes
const (
	A = iota + 1
	B
    _ // skip
	C
)
\end{lstlisting}

\end{paracol}

\section{TIC-80 with \WASM}

\noindent In order to use \WASM with the TIC-80, the TIC-80 executable must be compiled with \lstinline|-DBUILD_PRO=On|. The full setup proceedure is then
\\

\begin{lstlisting}[language=Bash]
git clone "https://github.com/nesbox/TIC-80" 
cd TIC-80/build
# run cmake
cmake .. -DBUILD_PRO=On
make -j4
# link binary
sudo ln -s $(pwd)/bin/tic80 /usr/local/bin/tic80
\end{lstlisting}

\noindent Refer to the \href{https://github.com/nesbox/TIC-80}{TIC-80 readme} for full installation instructions for your OS.

When executing a \lstinline|*.wasmp| script, we also require a setupfile with tile set, wave forms, etc. This is already included in the example repository (see \fullref{sec:tinygo-tic80}). We will generally use the CLI arguments \lstinline|--skip --fs .| to skip the startup animation and to mount the current working directory as the filesystem. 

We load and start the \WASM executable with\\

\begin{lstlisting}[language=Bash]
load wasmdemo.wasmp 
import binary cart.wasm 
run
\end{lstlisting}

\subsection{Specifications}
\noindent There is a full rundown of the TIC-80 and its functions \href{https://tic80.com/learn}{on the official site}. Another very useful resource is the \href{https://github.com/nesbox/TIC-80/wiki}{GitHub wiki}.
\\[1em]
\noindent Some general points:
\begin{itemize}
    \item TIC-80 runs at 60 fps.
\end{itemize}

\section{Using \tinygo to target TIC-80}\label{sec:tinygo-tic80}

We have an example repository, containing a Makefile for targetting TIC-80 with \tinygo in \href{https://github.com/fjebaker/global-game-jam-2023}{fjebaker/global-game-jam-2023}. It is also worth reading the \tinygo documentation on the \href{https://tinygo.org/docs/concepts/compiler-internals/differences-from-go/}{Differences from Go}. 

Finally, it is also worth noting that someone has already implemented a Go module for the TIC-80 \href{https://github.com/sorucoder/tic80/blob/master/tic80.go}{sorucoder/tic80} which we can use to base our implementation on. I am reluctant to just use this module as we won't really learn the memory map overview, and wrapping new functions is not too difficult.


\subsection{Setup}

\tinygo requires all sorts of setup configuration, but is probably easiest to use directly from the Docker image provided by the maintainers. As long as you have a Docker runtime installed and running, the Makefile included in the example repository should work fine.

Inline with the restrictions of the TIC-80, we have a \lstinline|target.json| file included which sets up the memory topology and linker flags needed. Interesting to note is that although we \textit{should not need} to provide an entry point, go \textbf{still requires} that the \lstinline|_start| \textbf{symbol is invoke} to setup the garbage collector and thread runtime. We can delegate this task to the \lstinline|BOOT| function.
\\
\begin{lstlisting}
//go:export BOOT
func BOOT() {
    tic80.Init()
}
// still need this since _start calls main
func main() {}
\end{lstlisting}

\subsection{Modifying \texttt{tic80.go}}

The full memory map of the TIC-80 is probably easiest to understand by looking at the C example \href{https://github.com/nesbox/TIC-80/blob/main/templates/c/src/tic80.h}{here}. We add new functions simply with the compiler directives. For example, the \lstinline|print| function might look like:
\\
\begin{lstlisting}
import unsafe

//go:export print
func print(textBuffer unsafe.Pointer, x, y int32, color, fixed, scale, alt int8) int32
\end{lstlisting}

\end{document}